\documentclass[11pt, a4paper]{article}
%\usepackage{geometry}
\usepackage[inner=2cm,outer=2cm,top=2.5cm,bottom=2.5cm]{geometry}
\pagestyle{empty}
\usepackage{graphicx}
\usepackage{fancyhdr, lastpage, bbding, pmboxdraw}
\usepackage[usenames,dvipsnames]{color}
\definecolor{darkblue}{rgb}{0,0,.6}
\definecolor{darkred}{rgb}{.7,0,0}
\definecolor{darkgreen}{rgb}{0,.6,0}
\definecolor{red}{rgb}{.98,0,0}
\usepackage[colorlinks,pagebackref,pdfusetitle,urlcolor=darkblue,citecolor=darkblue,linkcolor=darkred,bookmarksnumbered,plainpages=false]{hyperref}
\renewcommand{\thefootnote}{\fnsymbol{footnote}}

\pagestyle{fancyplain}
\fancyhf{}
\lhead{ \fancyplain{}{\textsc{IDS 702:\ Modeling and Representation of Data}} }
%\chead{ \fancyplain{}{} }
\rhead{ \fancyplain{}{\textsc{Fall 2019}} }
\rfoot{Page \thepage}
%\fancyfoot[RO, LE] {page \thepage\ of \pageref{LastPage} }
\thispagestyle{plain}

%%%%%%%%%%%% LISTING %%%
\usepackage{listings}
\usepackage{caption}
\DeclareCaptionFont{white}{\color{white}}
\DeclareCaptionFormat{listing}{\colorbox{gray}{\parbox{\textwidth}{#1#2#3}}}
\captionsetup[lstlisting]{format=listing,labelfont=white,textfont=white}
\usepackage{verbatim} % used to display code
\usepackage{fancyvrb}
\usepackage{acronym}
\usepackage{amsthm}
\VerbatimFootnotes % Required, otherwise verbatim does not work in footnotes!



\definecolor{OliveGreen}{cmyk}{0.64,0,0.95,0.40}
\definecolor{CadetBlue}{cmyk}{0.62,0.57,0.23,0}
\definecolor{lightlightgray}{gray}{0.93}


\lstset{
%language=bash,                          % Code langugage
basicstyle=\ttfamily,                   % Code font, Examples: \footnotesize, \ttfamily
keywordstyle=\color{OliveGreen},        % Keywords font ('*' = uppercase)
commentstyle=\color{gray},              % Comments font
numbers=left,                           % Line nums position
numberstyle=\tiny,                      % Line-numbers fonts
stepnumber=1,                           % Step between two line-numbers
numbersep=5pt,                          % How far are line-numbers from code
backgroundcolor=\color{lightlightgray}, % Choose background color
frame=none,                             % A frame around the code
tabsize=2,                              % Default tab size
captionpos=t,                           % Caption-position = bottom
breaklines=true,                        % Automatic line breaking?
breakatwhitespace=false,                % Automatic breaks only at whitespace?
showspaces=false,                       % Dont make spaces visible
showtabs=false,                         % Dont make tabls visible
columns=flexible,                       % Column format
morekeywords={__global__, __device__},  % CUDA specific keywords
}



\usepackage{array}
\newcolumntype{L}[1]{>{\raggedright\arraybackslash}p{#1}}
\usepackage{enumitem}
\usepackage{booktabs}
\usepackage{makecell}




%%%%%%%%%%%%%%%%%%%%%%%%%%%%%%%%%%%%
\begin{document}
\renewcommand{\arraystretch}{1.5}	


\begin{center}
{\Large \textsc{IDS 702:\ Modeling and Representation of Data}}
\end{center}


\begin{center}
	\textsc{Fall 2019} \\
	\textsc{Duke University} \\
\end{center}



\begin{center}
\begin{minipage}[t]{.9\textwidth}
\begin{tabular}{@{}L{3cm}L{12cm}}
	\toprule[0.065cm]
\textsc{Instructor:} & \href{https://akandelanre.github.io.}{\textsc{Olanrewaju M. Akande, Ph.D.}} \\
\textsc{Email:} &\href{mailto:olanrewaju.akande@duke.edu}{\Envelope ~olanrewaju.akande@duke.edu} \\
\textsc{Office:} & 256 Gross Hall \\
\textsc{Office Hours:} & Mon/Tue 3:00 - 4:00pm, 256 Gross Hall. Other hours available by appointment. \\
\textsc{Teaching Assistants:} & \href{https://datascience.duke.edu/lidia-azu-azucena-morales-vasquez}{Azucena Morales}. Office hours: Thur 2:30 - 4:30pm, 257 Gross Hall.
\newline Chenxi Wu. Office hours: Wed 3:00 - 5:00pm, 257 Gross Hall. \\ 
\textsc{Lectures:} & Tue/Thur, 10:05 - 11:20am, 270 Gross Hall. \\
\textsc{Labs:} & Every other Fri (starting Sep 6), 10:00 - 11:45am, 270 Gross Hall.\\
\textsc{Course Page:} & \href{https://akandelanre.github.io/IDS702_F19/syllabus/}{https://akandelanre.github.io/IDS702\_F19/} \\
\textsc{Recommended Textbooks:} & \href{https://www.amazon.com/gp/product/052168689X/ref=as_li_qf_sp_asin_il_tl?ie=UTF8&camp=1789&creative=9325&creativeASIN=052168689X&linkCode=as2&tag=andrsblog0f-20&linkId=PX5B5V6ZPCT2UIYV}{\textit{Data Analysis Using Regression and Multilevel/Hierarchical Models}} by Gelman A., and Hill, J.\\
\textsc{Optional Textbooks:}	& \href{http://faculty.marshall.usc.edu/gareth-james/ISL/}{\textit{An Introduction to Statistical Learning with Applications in R}} by James, G., Witten, D., Hastie, T., and Tibshirani, R. (Free pdf available online via the link.) \\
\textsc{Important Dates:} & Monday, September 2 -- Labor day; classes in session \newline Fri, September 6 -- Drop/add ends \newline Fri, October 4, 7:30pm -- Fall break begins \newline Wed, October 9, 8:30am -- Fall break ends \newline Mon, November 4, 11:59pm -- Final project proposal due \newline Tue, November 26, 10:30pm -- Thanksgiving; graduate classes end \newline Tue, December 3 and Thur, December 5 -- Final project presentations \\
	 \bottomrule[0.065cm]
\end{tabular}
\end{minipage}
\end{center}


\vspace{.5cm}
\setlength{\unitlength}{1in}
\renewcommand{\arraystretch}{1.5}



\section{Course Overview}
Statistical models are necessary for analyzing the type of multivariate (often large) datasets that are usually encountered in data science and statistical science. This is a graduate level course, within the curriculum for Duke's Master in Interdisciplinary Data Science (MIDS) program, that aims to provide students with the statistical data analysis tools needed to succeed as data scientists. 

In this course, you will learn the general work flow for building statistical models and using them to answer inferential questions. You will learn several parametric modeling techniques such as generalized linear models, models for multilevel data and time series models. You will also learn to handle messy data, including data with missing or erroneous values, and data with outliers or non-standard distributions. You will be able to assess model fit, validate model assumptions and more generally, check whether proposed statistical models are appropriate for any given data. You will also learn causal inference under the potential outcomes framework. Should time permit, we may also briefly cover nonparametric models such as classification and regression trees. 

Although this course emphasizes data analysis over rigorous mathematical theory, students who wish to explore the mathematical theory in more detail than what is covered in class are welcome to engage with and request further reading materials from the instructor outside of class.


\section{Learning Objectives}
By the end of this course, students should be able to
\begin{itemize}[label= {\color{darkblue}{\ArrowBoldRightStrobe}}]
	\item Use the statistical methods and models covered in class to analyze real multivariate data that intersect with various fields. 
	\item Assess the adequacy of statistical models to any given data and make a decision on what to do in cases when certain models are not appropriate for a given dataset.
	\item Cleanup and analyze messy datasets using approaches covered in class.
	\item Hone collaborative and presentations skills through the process of consistent team work on and class presentations of team projects.
\end{itemize}



\section{Prerequisites}
Students are expected to know all topics covered in the MIDS summer course review and boot camp. These include basic statistical inference including hypothesis testing, confidence intervals, linear regression with one predictor, and exploratory data analysis methods. Students are also expected to be familiar with \textsf{R}. Due to space constraints, the course is open only to students in the MIDS program.


\section{Class Materials}
Lecture notes and slides, lab exercises and other reading resources will be posted on the course website. We will only loosely follow the textbooks.


\section{Graded Work} 
Graded work for the course will consist of methods and data analysis assignments, team projects, and a final project. 
\begin{itemize}[label= {\color{darkblue}{\ArrowBoldRightStrobe}}]
	\item There is no final exam. Students' final grades will be determined as follows:
	\begin{table}[h]
		\centering
		\begin{tabular}{ll}
			Component & Percentage \\ \hline
			Methods and Data Analysis Assignments & 30\% \\
			Final Project & 30\% \\ 
			Team Project 1 & 15\% \\
			Team Project 2 & 15\% \\
			Lab Assignments & 10\% \\ \hline 
		\end{tabular}
	\end{table}
	
	\item There are no make-ups for assignments or the projects except for medical or familial emergencies or for reasons approved by the instructor before the due date. See the instructor in advance of relevant due dates to discuss possible alternatives. 
	
	\item Grades may be curved at the end of the semester. Cumulative averages of 90\% -- 100\% are guaranteed at least an A-, 80\% -- 89\% at least a B-, and 70\% -- 79\% at least a C-, however the exact ranges for letter grades will be determined after the final exam. 
\end{itemize}


\section{Descriptions of graded work}
\subsection{Methods and Data Analysis Assignments}
Methods and Data Analysis assignments are posted on the IDS 702 course website. Students turn in these assignments at the beginning of class on the due date. Students are permitted to work with others on the assignments, but each person must write up and turn in her or his own answers. The Methods and Data Analysis assignments include questions on the computational and the mathematical aspects of the methods that underpin the statistical models we learn during the semester, and questions that ask students to apply the modeling skills discussed during the semester. The assignments must be typed up using \textsf{R} Markdown, \LaTeX or another word processor, and submitted on \href{https://www.gradescope.com/courses/57701/assignments}{Gradescope} under Assignments. Note that you will not be able to make online submissions after the due date, so be sure to submit before or by the Gradescope-specified deadline.

\subsection{Lab Assignments}
The objective of the lab assignments is to give you more hands-on experience with data analysis using R. The labs times also gives you an additional platform to ask for help for your team and individual projects. Lab attendance is not mandatory on days when team presentations will not hold, however, each lab assignment should be submitted in timely fashion on the due date. You are REQUIRED to use R Markdown to type up your lab reports.

\subsection{Team Projects}
For the team projects, students work in teams to analyze data selected by the instructor. Students write a report with their data analysis findings. Students may have the opportunity to present their results in class. Detailed instructions will be made available later on the website.

\subsection{Final Project}
For the final project, students analyze a data-based research question of their choosing, subject to the instructor's approval. The data should comprise several variables amenable to statistical analyses via modeling. Students can bring in their own research data sets, or they can ask the instructor for assistance with identifying appropriate data. Students present their results in a poster session. Detailed instructions will be made available later on the website.


\section{Late Submission Policy} 
\begin{itemize}
	\item  You will lose 50\% of the total points on each homework if you submit within the first 24 hours after it is due, and 100\% of the total points if you submit later than that.
	\item You will lose 40\% of the total points on each lab if you submit within the first 24 hours after it is due, and 100\% of the total points if you submit later than that.
\end{itemize}


\section{Tentative Course Schedule} 
We will cover the topics below. We may spend different amounts of time on each topic, depending on the interests of students. For a detailed and updated outline, check on the updated course schedule on the course page regularly. 
\begin{enumerate}[label= {\color{darkblue}{\ArrowBoldRightStrobe}}]
	\item Introduction to course % \dotfill ~$\approx$ 1 lecture
	\item Linear regression % \dotfill ~$\approx$ 5 lectures
	\begin{enumerate}[label= {\color{cyan}{\Rectangle}}]
		\item Multiple predictors
		\item Inference and prediction
		\item Model assessment, diagnostics and validation
			\item Transformations, multicollinearity and heteroscedasticity
		\item Model building and selection
	\end{enumerate}
	\item Logistic regression
	\begin{enumerate}[label= {\color{cyan}{\Rectangle}}]
		\item Interpreting logistic regression coefficients
		\item Inference vs prediction
		\item Model assessment and validation
	\end{enumerate}
	\item Generalized linear models
	\begin{enumerate}[label= {\color{cyan}{\Rectangle}}]
		\item Multinomial logistic
		\item Poisson regression
		\item Probit regression
	\end{enumerate}
	\item Introduction to multilevel models
	\begin{enumerate}[label= {\color{cyan}{\Rectangle}}]
		\item Fixed effects vs random effects
		\item Multilevel linear models
		\item Multilevel logistic regression
	\end{enumerate}
	\item Dealing with messy data
	\begin{enumerate}[label= {\color{cyan}{\Rectangle}}]
		\item Missing values, errors, and outliers
		\item Single imputation methods
		\item Multiple imputation
	\end{enumerate}
	\item Methods for causal inference
	\begin{enumerate}[label= {\color{cyan}{\Rectangle}}]
		\item Association vs. causation, confounding, and Simpson's paradox.
		\item The potential outcome framework: potential outcomes, assignment mechanism and estimands.
		\item Observational studies with ignorable assignment mechanism and propensity scores: assumptions of ignorability (unconfoundedness), matching, weighting, regression, and propensity scores methods. 
	\end{enumerate}
	\item Time series models
	\item Wrap up (would cover introduction to nonparametric methods should time permit).
\end{enumerate}


\section{Academic Integrity}  
Duke University is a community dedicated to scholarship, leadership, and service and to the principles of honesty, fairness, respect, and accountability. Citizens of this community commit to reflect upon and uphold these principles in all academic and nonacademic endeavors, and to protect and promote a culture of integrity. To uphold the \href{https://studentaffairs.duke.edu/conduct/about-us/duke-community-standard}{Duke Community Standard}:
\begin{itemize}[label= {\color{darkred}{\Large \HandRight}}]
	\item I will not lie, cheat, or steal in my academic endeavors;
	\item I will conduct myself honorably in all my endeavors; and
	\item I will act if the Standard is compromised.
\end{itemize}

Cheating on exams or plagiarism on homework assignments, lying about an illness or absence and other forms of academic dishonesty are a breach of trust with classmates and faculty, violate the Duke Community Standard, and will not be tolerated. Such incidences will result in a 0 grade for all parties involved. Additionally, there may be penalties to your final class grade along with being reported to the Undergraduate Conduct Board. Please review the academic dishonesty policies at \url{https://studentaffairs.duke.edu/conduct/z-policies/academic-dishonesty}.


\section{Diversity \& Inclusiveness:}
In line with the MIDS culture, this course is designed so that students from all backgrounds and perspectives all feel welcome both in and out of class. Please feel free to talk to me (in person or via email) if you do not feel well-served by any aspect of this class, or if some aspect of class is not welcoming or accessible to you. My goal is for you to succeed in this course, therefore, please let me know immediately if you feel you are struggling with any part of the course more than you know how to manage. Doing so will not affect your grades, but it will allow me to provide the resources to help you succeed in the course.


\section{Disability Statement} 
Students with disabilities who believe that they may need accommodations in the class are encouraged to contact the \href{https://access.duke.edu/students/staff.php}{Student Disabilities Access Office} at 919.668.1267 or \href{mailto:disabilities@aas.duke.edu}{disabilities@aas.duke.edu}  as soon as possible to better ensure that such accommodations are implemented in a timely fashion.


\section{Other Information} 
It can be a lot more pleasant oftentimes to get in-person answers and help. Make use of the teaching team's office hours, we're here to help! Do not hesitate to come to my office during office hours or by appointment to discuss a homework problem or any aspect of the course.  Questions related to course assignments and honesty policy should be directed to me. When the teaching team has announcements for you we will send an email to your Duke email address. Please make sure to check your email daily.


\section{Professionalism}
Please refrain from texting or using your computer for anything other than coursework during class.

\end{document} 